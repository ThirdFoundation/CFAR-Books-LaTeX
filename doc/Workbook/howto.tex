\clearpage
\chapter*{How to use this book}
Welcome, participant!

This is your workbook for the next four days.  You'll want to bring it with you to classes, conversations, and anywhere else where you feel like taking notes.  Inside, you'll find prompts, handouts, diagrams, and useful questions, as well as plenty of space for your own thoughts and a glossary for quick reference.  You \emph{won't} find in-depth descriptions of the CFAR material---we've found that our classes work best when our participants are fully engaged, rather than reading along, and so we've printed a separate handbook with explanations, theoretical underpinnings, and further resources that you can dig through between sessions and after the workshop.

\textbf{Two opposite pieces of advice to reconcile, before you dive in:}

First, remember that this workbook is a tool.  It's a helpful guide, not a set of homework problems.  You may find yourself eagerly filling out some pages while finding others unhelpful or irrelevant.  Trust yourself, and use the pages within in ways that make sense to you.

Second, don't forget to try things!  If you're \emph{not} sure whether a given prompt or activity will be valuable, the best way to find out is to give it a shot.  Don't be afraid to spend a few minutes scribbling as part of a productive experiment---it's often hard to see the value of a technique in the abstract, and easy to miss opportunities for growth when you default to old habits or inaction.  If you find even a single new tool for solving problems in these pages, it'll be well worth the time you spent ruling out all of the others.

Best of luck, and may your workshop be full of interesting conversations!
\clearpage



