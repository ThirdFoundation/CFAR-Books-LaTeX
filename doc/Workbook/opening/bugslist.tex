\setlength{\parindent}{0em}
\emph{Our summary of the information from this session can be found in your handbook on p. 999}

\section*{What is a ``bug''?}
\setlength{\parindent}{1.5em}
A bug is anything that systematically goes wrong in your life---any negative emotion or outcome that emerges as a side effect of your current habits, beliefs, or ways of being.  These can range from trivial annoyances (like never being able to find the matching sock) to major issues with health, work, relationships, and life satisfaction.  While some bugs are unavoidable, most can be ameliorated or solved entirely, and regular ``debugging'' is a straightforward and effective way to make improvements in your life.

\section*{Some common examples of bugs}
\begin{itemize}
\item Forgetting to answer important calls or emails
\item Never being able to talk to a certain person without ending up in an argument
\item Constantly remembering that you need batteries, but never while you're actually at a store
\item Starting important assignments right before the deadline, despite hating all-nighters
\item Feeling that the work you're doing isn't relevant or important, even though you're good at it and enjoy it day-to-day
\item Being disproportionately irritated or frustrated by little things a coworker is doing that ``shouldn't'' bother you
\item Always being a day or a week or a month away from taking that vacation or starting that project, no matter how much time passes
\item Bingeing on television or sweets or hobbies or Reddit
\item Wanting to want something (like exercise or keeping up with old friends) that you never really want in the moment
\end{itemize}


\section*{Making a list}
The best way to identify your bugs is to notice them \emph{as they occur.}  Flashes of annoyance, feelings of conflict, trouble making decisions or staying focused, a sense that you're burning energy or ``making'' yourself do things---all of these are convenient flags where some part of your psyche is letting you know that what you're currently doing isn't \emph{quite} right.  If you have any sort of system for regular note-taking (such as a notes tab on your phone or computer, or a notebook that you carry around with you), you may want to add a section for recording bugs on the fly.

There are also some simple strategies for identifying bugs retroactively.  One common method is to \textbf{get an echo}---try saying, out loud, ``Everything went perfectly last week!'' and then listen to your own thoughts to see what bubbles up.  Another method is to do a \textbf{mental walk-through} of your average day or week, noting all of the things that bother you about your morning routine, your commute, your workflow, your evenings.  You can also run down the list of major life domains---how is your health?  Your finances?  Your relationships?  Your friends and family?  Your job?  Your to-do list?  

\setlength{\parindent}{0em}
\includegraphics[width=\textwidth]{../../../img/linesquarter.png}
\setlength{\parindent}{1.5em}

As you start to slow down, you can try and key into things you \textbf{said you'd do but haven't}, things you \textbf{tried to do but failed}, times you \textbf{felt sure but were wrong}, words you \textbf{wish you could take back} (and the thing that caused you to say them), time and money you \textbf{wish you'd spent differently}, and anything that you'd feel a little squidgy about sharing with your heroes and role models and people you'd like to impress.

\setlength{\parindent}{0em}
\includegraphics[width=\textwidth]{../../../img/linesquarter.png}
\setlength{\parindent}{1.5em}

\section*{Effective debugging}
Not all bugs are created equal---some are large and complex, some are simple but have a solution that requires lots of time and effort, some are private, some are trivial.  Now that you have an initial list, see if you can create some sort of tentative ranking or structure, so that you'll always have a ``next bug'' to work on throughout the weekend.  Questions to consider:

\begin{itemize}
\item Which of these feels the most significant?
\item Do any of these need to be solved before I can make progress on the others? 
\item Are some of these better suited to working with partners or groups?
\item Which of these will take more than ten minutes to solve?  Which will take less?
\item Which ones am I completely stuck on?  Which ones do I already have ideas for?
\end{itemize}

\setlength{\parindent}{0em}
\includegraphics[width=\textwidth]{../../../img/lineshalf.png}
\setlength{\parindent}{1.5em}

Don't forget---this is a preliminary list!  As you progress through the workshop, you should feel free to add to it, reorganize it, and knock things off it.  Remember---there's no ceiling on how awesome things can get, so if you run out of negative bugs to fix, start looking for ``positive bugs'' in the form of opportunities you haven't fully seized.  When it comes to bugs lists, \emph{finished} is just shorthand for \emph{my next bug is that I can't think of any more bugs.}