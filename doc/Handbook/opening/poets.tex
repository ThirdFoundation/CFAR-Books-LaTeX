\section*{``Bad poets borrow---good poets \emph{steal}.''}

If you set out to write a poem, you may find that a particular turn of phrase from another poem just won't stop popping into your thoughts.  Somehow, Homer's reference to the sun as \emph{extending rosy fingers} is exactly what you're trying to say, and you're tempted to just drop it in and call it an homage.

\setlength{\parindent}{1.5em}
But a funny thing is likely to happen here if you actually use Homer's quote (especially if you found yourself borrowing from other sources, too).  There's a certain kind of smoothness and cohesion that's going to be missing---an irregularity of style and pacing that makes your poem come across as a collection of fragments instead of a single unified whole.  It might be a \emph{nice} poem, but it won't be as vivid, as \emph{true,} as what you hoped to convey.

By way of contrast, you could steal the \emph{source} of Homer's thinking---give yourself the same input, rather than simply mimicking his output.  Put yourself in Homer's sandals---how did he come to produce that particular illustration of a dawning day?  What way of looking at the world, and of experiencing language, allowed him to come up with that exact expression?  We only get to see Homer's work, not his actual process, but if you try, you can peer behind the words, and get a glimpse of the source that generated them---of a \emph{way of thinking} that you can emulate to create your own original beauty.

In a similar way, applied rationality as it is presented in this book is the output of other minds.  It's a product you can't \emph{quite} steal for yourself---if you try, you'll find that it never quite fits.  The techniques are excellent algorithms for improving thinking and solving problems, but they aren't the heart and soul of rationality, any more than rosy fingers are the essence of Homer's genius.  They're a jumping-off point, and what you want to jump \emph{toward} is the place the techniques came from---the particular way of thinking about the world and about ourselves that led to someone sitting down and inventing the contents of this book.

This is the real goal of your practice.  As you work through the techniques, try to see this underlying essence.  Aim for a combination of openness and hubris---that is, act like a person who has something to learn, but \emph{also} like a person who's already in possession of a lifetime of knowledge and experience.  Assume that you have joint ownership over your own rationality education, and over the subject of applied rationality as a whole.  We find that the people who get the most out of our workshops are the ones who are simultaneously curious and confident---they're the ones who walk away with all of our techniques \emph{and} the ability to create new techniques of their own.


