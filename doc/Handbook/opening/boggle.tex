\clearpage
\section*{Boggle!}

There's a way in which education tends to make knowledge very \emph{flat}.

Let's take the Earth and the Sun, for example.  If I were to ask you about the relationship between the two, you'd probably offer me the well-worn phrase ``the Earth revolves around the Sun.''  It's automatic, reflexive, almost atomic---once you start with ``the Earth,'' you barely have to think anymore.  The ``revolves around the Sun'' part just fills itself in.

But once upon a time, people didn't \emph{know} that the Earth revolved around the Sun.  In fact, people didn't even really know what the Earth and the Sun \emph{were}---they thought they did, but looks can be deceiving.  It took us multiple geniuses and the innovations of centuries to go from ``the Earth is a flat plane and the Sun travels across the celestial sphere'' to the factoid that we repeat back to our teachers in a bored monotone.  Somehow, all of the confusion and excitement of discovering that the Sun is an incandescent ball of hydrogen and that the Earth is tied to it by the same fundamental force which makes pendulums swing and that both of them are round except \emph{not quite} and that gravitational attraction is proportional to the square of the distance except \emph{not quite}, don't forget relativity and quantum mechanics and---

---somehow, all of that gets lost when we flatten things out into ``the Earth revolves around the Sun.''

Fortunately, there's a solution---\emph{boggling.}  You're reading a book!  What's a book?  I mean, okay, it's just a book.  But what is it \emph{really?}  I mean, where did these pages come from?  Who wrote them?  Who manufactured them?  How could \emph{you} make a book?  I mean, maybe you've already made one.  But how did the paper get made?  And what's printer ink made of, anyway?  And where did the ideas come from?  And language!  These squiggles on a page carry \emph{meaning!}  How'd we come up with that?  What's actually going on in your brain, when you look at these squiggles and find yourself thinking thoughts?  What even \emph{is} a thought?  I hear there are neurons involved---how does \emph{that} work?

When you allow yourself to embrace confusion, and turn away from the cached, easy, empty answers, you start to see a much richer, deeper world, with many more opportunities to learn and to grow.  During the workshop, there will be many things that \emph{seem like} stuff that you already know, just as you already know that the Earth revolves around the Sun.  But don't be fooled!  Surface explanations are the opposite of knowledge---they're a curiosity-killer, preventing you from noticing that there's stuff you still don't \emph{get}.  Human cognition is one of the most complex, opaque, and difficult phenomena we've ever encountered.  As you study it, don't settle for flat knowledge---instead, \emph{boggle.}