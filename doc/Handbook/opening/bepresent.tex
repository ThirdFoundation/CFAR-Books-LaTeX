\clearpage
\section*{Be Present}

One key element of getting the most out of this workshop is being \emph{present} in a fairly general sense. This includes being physically present, but it also includes having your mind in the room and your background thoughts focused on the content. The more you're taking calls and answering texts and keeping up with social media and what's going on back home, the more you'll remain in your ordinary mental space, continuing to reinforce the same habits and patterns you're here to change.  There's a sort of snowball effect, where even a little disengagement can make absorbing the value you'd like from the workshop rather difficult.

In addition to external distraction, we've also found that there are a few unhelpful narratives that participants occasionally find themselves repeating---narratives which make it hard to engage with the content and block opportunities for asking good questions and taking new steps.  If you notice one of these narratives cropping up in the back of your mind, we encourage you to try deliberately setting it aside, as an experiment---let it go, see what happens, and judge for yourself.  Our staff are happy to chat with you about any of these, if you think you might find that helpful.

\begin{itemize}
	\item \textbf{``I'm too dumb/old/lazy to learn this."} We sometimes encounter people who think that, because they don't measure up to some standard or another, they aren't ``good enough'' to benefit from the workshop material.  As a counter to this, we recommend donning a \emph{growth mindset}: any capacity you currently lack is an opportunity for improvement and a sign pointing the way.  If it can be learned by a human, it can be learned by you.	
	\item \textbf{``I already know this part."} Some people come into our workshop with background knowledge about psychology, cognitive biases, or habit formation, and, when they start to see familiar material, slip into a mode of assuming there's nothing for them to learn.  Unfortunately, this can mean that you're ``turning off'' right at the moment that we're offering new and powerful insights.  To counter this, we recommend that you try to approach \emph{every} class with fresh eyes.  Even if the core concepts are familiar, look for the fine detail and the subtle shades of significance---the places where your peers and instructors have made valuable connections you might have missed.  In particular, try to be interested rather than interesting---if you're too busy sharing your own knowledge, you may overlook opportunities to steal productively from others.
	\item \textbf{``I've got important things to do, and this lesson can wait."} Sometimes there really \emph{are} important things to attend to. But if they're on your mind during the workshop, you're likely to have a hard time absorbing the material in a way that will stick. We recommend that you set aside what you can, and \emph{fully address} what you can't set aside: if something really can't wait, step out, make it your sole focus until it's dealt with, and return with full and fresh attention.
\end{itemize}
