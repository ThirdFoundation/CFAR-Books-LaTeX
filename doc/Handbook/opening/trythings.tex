\clearpage
\section*{Try Things!}

When you're considering adopting new habits or ideas, there's no better way to gather data than \emph{actually trying.}  It's often faster and simpler to just give things a shot and see how it goes than to spend a lot of time trying to anticipate and predict whether or not you'll find something worthwhile.

This is particularly important because when something \emph{does} work out, \emph{you get to keep doing it.}  If your friends have recommended five different activities to you, and you've only liked one of them, it's easy to think of the whole process as a pretty big waste of time:

\setlist[description]{leftmargin=4.5cm,labelindent=4.5cm}
\begin{description}
	\item[\ding{55}] Yoga
	\item[\ding{55}] Ultimate Frisbee
	\item[\ding{55}] Dungeons \& Dragons
	\item[\ding{52}] Meditation
	\item[\ding{55}] Salsa dancing
\end{description}

An 80\% failure rate isn't exactly encouraging, after all.  But what the above framing fails to take into account is the magnitude of even a single success.  Instead of four bad experiences and one good one, what's \emph{actually} going on is more like the following:

\begin{center}
  \begin{tabular}{ | l | c | c | c | c | c | c | c | c | c |}
    \hline
    Activity & T1 & T2 & T3 & T4 & T5 & T6 & T7 & T8 & T9 \\ \hline
    Yoga & \ding{55} & \ding{55} & & & & & & & \\ \hline
    Ultimate Frisbee & \ding{55} & & & & & & & &  \\ \hline
    Dungeons \& Dragons & \ding{55} & \ding{55} & \ding{55} & & & & & &  \\ \hline
    Meditation & \ding{52} & \ding{52} & \ding{52} & \ding{52} & \ding{52} & \ding{52} & \ding{52} & \ding{52} & \ding{52} \\ \hline
    Salsa dancing & \ding{55}  & & & & & & & & \\ \hline
    \end{tabular}
\end{center}

When you look at it this way, you can see that the failed trials are more than compensated for by the sustained run of a now-successful habit.  Indeed, when it comes to hobbies and activities that might last you the rest of your life, it becomes worthwhile to establish a habit of trying things that have even a one-in-ten or one-in-a-hundred chance of being enjoyable.  It only takes a few paying off to make the whole thing worthwhile.

So while you're listening and participating this weekend, be on the lookout for opportunities to turn our lessons into actions that you can actually try out, right then and there.  Translating class material into practical experiments is a great way to digest material anyway, and it'll help you decide which techniques are most worth prioritizing when you return home.