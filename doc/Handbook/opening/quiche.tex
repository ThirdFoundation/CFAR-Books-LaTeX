\section*{Make good quiche}

Imagine that you have a friend who is creating a recipe book.  You've agreed to help your friend beta test some of their recipes, and they've handed you a rough draft of instructions on how to make quiche.

As you're reading through the recipe, you begin to notice a few ... let's say, \emph{problematic} steps.  For instance, the recipe calls for six ``whole eggs,'' which to you seems to imply shells and all.  It also says to bake for 4.5 hours at 450 degrees, and calls for 10 tablespoons of salt.

Now, one way that you might offer productive feedback to your friend is to follow the recipe \emph{exactly as written,} creating a crunchy, salty, burned quiche.  This is actually a pretty helpful strategy, early on---it's a way to stress-test the recipe to see exactly how broken it is.

However, if you \emph{also happen to want some quiche}, there's another method you might employ.  Instead of following steps that are obviously wrong, you could instead \emph{try to make good quiche}, treating the recipe as more of an inspiration than a strict set of instructions.  You could throw away the eggshells, drop the time and temperature down to (say) 45 minutes at 350 degrees, and throw in just a pinch of salt.  Maybe you'll even have some additions that your friend didn't think of, like mixing in some chopped kale.

At the end of \emph{that} process, you'll not only have notes about flaws in the original recipe, but also constructive suggestions and---most importantly---a delicious meal you can actually eat.  You'll have something that's useful to \emph{you}, both in the moment and for the future.

Like your friend's quiche recipe, many of the concepts and techniques within the workshop are experimental.  There will be times when they seem a little off, and other times when they may seem clearly false.  It helps to remember that the goal is not to improve our recipe book, but to \textbf{make good quiche}.  That means that, instead of doing things that don't make sense, you should feel free to tinker, experiment, and modify.  Your perspective is unique---while we have a lot of insight to offer, there's no one who better understands your own life and mind than you.  If we seem to be pointing in the wrong direction, feel free to head in the right one, instead---and afterward, let us know what you discovered.