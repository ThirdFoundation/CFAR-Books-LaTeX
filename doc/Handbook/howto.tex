\clearpage
\chapter*{How to use this book}
Welcome, participant!

This is your content handbook.  It's meant as a reference material, so that you can examine the theoretical underpinnings of our techniques, refresh yourself on particular steps or methods, and go beyond what's taught at the workshop.  Each technique has its own section, including:

\begin{itemize}
\item \textbf{Epistemic status} --- This is a measure of how closely linked to established cognitive science research a given technique is (versus those we have discovered through practical iteration, but not yet gathered formal data on).
\item \textbf{Step-by-step breakdown} --- While we encourage participants to mix, match, refactor, and tinker, we've distilled our basic algorithms into their most useful forms.
\item \textbf{Common mistakes and FAQ}
\item \textbf{Resources for further exploration}

While you may be tempted to read ahead, be forewarned---we've often found that participants not only have an easier time understanding our content if they hear it first, but also have a \emph{harder} time grasping it if they've already tried to put it together from the text.  Many of the explanations are intentionally approximate or incomplete, and most of them rely on terminology and context that's best transmitted in person.  It helps to think of this handbook as a companion to the workshop, rather than as a standalone resource.

\textbf{You do \emph{not} need this handbook during our classes;} your workbook contains all of the information and prompts you will need for our lectures and activities.  That being said, we hope that, between classes and after the workshop, this handbook provides you with lots of food for thought.  Don't be afraid to scribble notes in the margins and jot down your own annotations, and let us know if you find yourself discovering connections we haven't thought of!
\clearpage



