	\section*{Aiming to answer the right questions \ldots}
	Imagine that an intelligent, curious person meets with you for lunch, and begins asking you about your field of expertise.  At first, you talk mainly about your own day-to-day work, but eventually the conversation turns to the big, open questions---what are the most important unsolved problems in your profession?  Why do they matter?  What kinds of things change once you and your colleagues break through?  After you share your perspective, your new acquaintance nods thoughtfully, and asks one final question:
	
	``So, why aren't you working directly on \emph{that}?"
	
	There's a useful (and somewhat friendlier) generalization of this way of thinking.  At any given time in our lives, it's possible (though not always easy!) to answer the question, ``What is the \emph{most} important problem here, and what are the things that are keeping me from working on it?"  We refer to this as ``asking the Hamming question," as a nod to mathematician Richard Hamming, who was notorious for doing the above with his colleagues at Bell Laboratories.
	
	\section*{\ldots while accounting for our cognitive imperfections.}
	When we make decisions and analyze information, we tend to move back and forth between two broad kinds of thought---one faster, more automatic, and more closely tied to our emotions, and the other slower, more effortful, and more closely tied with our explicit thoughts and beliefs.  In his book \emph{Thinking Fast and Slow}, Nobel Prize winner Daniel Kahneman described them as \textbf{System 1} and  \textbf{System 2}:
	
	\renewcommand{\arraystretch}{1.2}
	\begin{center}
	\begin{tabular}{ |c|c| }
	\hline
	\textbf{System 1} &  \textbf{System 2} \\
	\hline \hline
	 Came first in our evolutionary history & Developed later and is more unique to humans \\
	 Wordless, ``black box'' thinking & Verbal, ``transparent" thinking \\
	 Processes information quickly & Processes information slowly \\
	 ``Intuition,'' ``instinct,'' ``reflex,'' etc. & ``Analysis,'' ``concentration,'' ``reflection,'' etc. \\
	 Always on, doesn't use working memory & Often on standby, limited by working memory \\
	 \hline
	 \end{tabular}
	 \end{center}
	 
	 Each ``system'' is complex, made up of a variety of parts, and neither is perfect.  Our knee-jerk, automatic processes are prone to making the wrong connections---if a new acquaintance resembles an old enemy, you may find yourself feeling anxious or cold without really knowing why.  Our deliberate, explicit processes can fail by leaving out information---if you can't put a fleeting feeling of unease into words, you may be tempted to disregard it, and exclude it from your calculations.
	 
	 Often, people make the mistake of thinking that rationality is the process of muting those primitive, intuitive processes and just relying on System 2.  It's an understandable mistake---after all, those are the ``higher brain'' functions, the ones that allow us to do things animals can't, like writing and philosophy and math and science.
	 
	 But turning off or ignoring large parts of your brain is rarely helpful, and applied rationality is about using every tool in your possession.  In the classes at this workshop, we'll talk about how to balance these two types of thinking, learning to understand the strengths of each so that you know when to bring them to bear and how to use them effectively both together and apart.  The aim is to make deliberate, thoughtful use of your \emph{whole} mind---a whole that's much greater than the sum of its parts.

\begin{center}
\includegraphics[width=\textwidth]{../../../img/lineshalf.png}
\end{center}