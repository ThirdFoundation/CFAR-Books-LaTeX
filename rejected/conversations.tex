\clearpage
\section*{Choose Your Conversations}

A significant part of the value of this workshop comes from the opportunity to have conversations that are very different from the ones you have in your everyday life.  You're at a rationality workshop, surrounded by interesting people you may not see again for a long time, if ever---how can you make the most of every minute?

One suggestion is to talk about things that \emph{actually matter to you.}  What is most important in your life right now?  What changes would you like to see in yourself and the world around you?  Where are you hungry for second opinions, new perspectives, and outside views?  You're much more likely to see positive change in an aspect of your life that you bring to light and make visible to others than one that you think about all by yourself.

Another suggestion is to \emph{refuse to be bored.}  When you start to feel yourself nodding lifelessly, stop; when you notice yourself wishing you were talking about something else, interrupt.  You only have four short days, and there are dozens of people worth talking to---here more than anywhere else, you should feel free to bend social rules that aren't pulling their weight, and be a little more direct about getting to the good stuff quickly.

\subsubsection{Signs that you're heading in the right direction:}
\begin{itemize}
	\item You feel curious---there's a gap in your knowledge, and you want help filling it.
	\item You feel surprised---this is new info, and it challenges your previous assumptions.
	\item You feel conflicted---you feel internal pressure to say something, but you also feel yourself holding back from nervousness or uncertainty.
\end{itemize}

\subsubsection{Signs that it's time to try something different:}
\begin{itemize}
	\item You feel bored or distracted.
	\item You feel like you're ``being polite.''
	\item You feel like you've had this conversation before.
\end{itemize}

If you find yourself caught in a conversation that doesn't feel valuable, try following up on an earlier, more interesting thread, or seeing if you can skip ahead to the end.  Ask if you can change the topic, or ``go meta''---be (gently) honest about the lack of aliveness, and recruit your conversational partner to help you figure out how to fix it.  And don't be afraid to simply abandon ship---remember, you're here for \emph{you.}  The goal is to be interested, not interesting---the most important thing to protect is your own sense of curiosity and delight.